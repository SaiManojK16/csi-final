\documentclass[11pt,a4paper]{article}
\usepackage[utf8]{inputenc}
\usepackage[margin=0.75in]{geometry}
\usepackage{enumitem}
\usepackage{hyperref}
\hypersetup{colorlinks=true, urlcolor=blue, linkcolor=black}

\pagestyle{empty}
\setlength{\parindent}{0pt}
\setlength{\parskip}{6pt}

\begin{document}

\begin{flushleft}
\textbf{Sai Manoj Kartala} \\
Albany, NY 12203 \\
kartalasaimanoj@gmail.com \\
+1-518-941-0211 \\
\href{https://www.linkedin.com/in/saimanojkartala}{LinkedIn} | Portfolio
\end{flushleft}

\vspace{0.2in}

\today

\vspace{0.15in}

\begin{flushleft}
ITS Human Resources \\
Office of Information Technology Services \\
Empire State Plaza, Swan Street Building, Core 4, Floor 1 \\
Albany, NY 12220
\end{flushleft}

\vspace{0.15in}

\textbf{Re: Business Systems Analyst 2 (NY HELPS) - 10073 - Vacancy ID: 201563}

\vspace{0.15in}

Dear Hiring Manager,

I am writing to express my strong interest in the Business Systems Analyst 2 (NY HELPS) position (Vacancy ID: 201563) at the New York State Office of Information Technology Services. With a Master of Science in Computer Science from SUNY Albany, extensive experience in business analysis, requirements elicitation, and a proven track record of leading analysis initiatives and identifying solutions that improve system efficiency and effectiveness, I am well-qualified to lead systems analysis and design initiatives supporting multiple agencies' and ITS shared service business analysis needs.

\textbf{Requirements Elicitation Using Business Analysis Techniques:} As Entrepreneurial Lead in the NSF I-Corps program, I conducted 30+ customer discovery interviews using structured interview techniques to elicit requirements and validate product–market fit for AI/AR STEM platforms. I employed observation, brainstorming, and survey/questionnaire methods to gather comprehensive business requirements, demonstrating my ability to use business analysis techniques such as interviews, observation, and brainstorming as the foundation for solutions to organizational business needs. This experience directly aligns with using workshops, focus groups, interviews, observation, brainstorming, and surveys/questionnaires to elicit requirements.

\textbf{Solution Identification Using Business Analysis Techniques:} Throughout my career, I have used business analysis techniques to identify solutions aimed at improving the efficiency and effectiveness of systems, business processes, and products. I performed root cause analysis to identify system bottlenecks and implemented solutions that reduced load times by 40\%. I created business process models and workflow diagrams to analyze current processes and identify improvement opportunities. At Venture Starters, I analyzed sector data and performed fit-gap analysis to identify \$2M+ in potential opportunities, demonstrating my ability to use business analysis techniques such as Root Cause Analysis, Business Process Model, and Fit-Gap Analysis to identify solutions that fulfill business requirements.

\textbf{Comprehensive Requirements Documentation:} I have extensive experience describing in comprehensive written documents what systems, processes, or products must do to satisfy established business requirements. Throughout my projects, I have created, updated, and maintained documentation through the system development life cycle, including Business Requirement Documents (BRD), Use Cases, Impact and Feasibility Analysis, and Change Management Analysis. I developed detailed documentation for AI-powered platforms, data pipelines, and system architectures, ensuring clear communication of requirements and system functionality. My experience includes documenting workflow processes, data inputs and outputs, and overall program goals and objectives throughout the SDLC.

\textbf{Requirements Validation Throughout SDLC:} Throughout my career, I have validated requirements throughout the product/system development life cycle, including all changes to processes that would enable organizations to achieve their goals. I have worked closely with stakeholders to validate requirements, conducted review sessions, gathered feedback, and refined requirements based on stakeholder input. My experience includes validating requirements at each phase of the SDLC, ensuring alignment between business needs and technical solutions, and validating that process changes support organizational goals.

\textbf{Requirements Verification and Testing:} I have extensive experience verifying requirements throughout the SDLC to ensure they perform to required specifications and achieve design capabilities. I have developed test plans and scenarios, tested scenarios, reviewed test results, identified constraints and risks, and communicated findings with stakeholders. As a Graduate Student Assistant, I review and grade 50+ coding assignments weekly, performing comprehensive testing and validation. I have experience with functional testing, system testing, and regression testing, identifying constraints and risks, and communicating test results to stakeholders throughout the development process.

\textbf{Current State Analysis and Process Modeling:} I have experience analyzing the current or "as-is" state of business processes and functions. At MARVLS, LLC, I analyzed existing systems and processes to identify optimization opportunities, creating process models to understand current workflows. I have documented current state processes, identified inefficiencies, and developed recommendations for improvement, which is essential for analyzing the as-is state and creating process models.

\textbf{Future State Requirements Documentation:} I have experience documenting requirements for future or "to-be" state of IT systems and business processes. I have created comprehensive documentation describing desired system functionality, process improvements, and business objectives. My work includes developing specifications for new systems, documenting process improvements, and creating roadmaps for achieving future state objectives, which is essential for documenting to-be state requirements for critical projects.

\textbf{Impact Analysis and Change Management:} I have experience analyzing the impact of changes to IT system requirements or business processes and evaluating business analysis delivery timelines and priorities. I have assessed the impact of system modifications, identified dependencies, and evaluated implementation timelines. My experience includes performing impact analysis for system changes, assessing resource requirements, and prioritizing initiatives based on business value and feasibility.

\textbf{Stakeholder Engagement and Issue Resolution:} I have extensive experience working directly with stakeholders to resolve issues with design specification challenges. My role conducting customer discovery interviews and presenting to mentors and investors demonstrates my ability to engage with stakeholders, understand their concerns, and work collaboratively to resolve challenges. I have experience facilitating discussions, negotiating solutions, and ensuring stakeholder buy-in for design specifications.

\textbf{Resource Management and Testing:} I have experience managing resources involved with system testing and assessing the impact of errors on system function. As a Graduate Student Assistant, I manage grading resources and coordinate testing activities. I have experience coordinating testing efforts, managing test resources, and assessing the impact of issues on system functionality during and after implementation.

\textbf{Documentation and Training:} I have experience scheduling and assigning staff to write business-focused system documentation and train users on new IT system or business process functions. I have created comprehensive documentation and training materials, and have experience mentoring students in technical concepts. My experience includes developing user guides, training materials, and conducting training sessions, which is essential for this role.

\textbf{Leadership and Mentoring:} I have experience guiding and mentoring entry-level staff. As a Graduate Student Assistant, I mentor students in debugging and optimization strategies, providing guidance and support. My leadership experience in the NSF I-Corps program demonstrates my ability to lead teams and guide others in business analysis activities.

\textbf{Methodology and Standards Development:} I have experience developing and communicating methodology, standards, and templates. I have created documentation templates, established coding standards, and developed processes for consistent project delivery. My experience includes creating reusable templates and standards to promote consistency, which is essential for assisting Finance Division management in developing methodology, standards, and templates.

I am particularly excited about this opportunity to lead systems analysis and design initiatives for multiple agencies and ITS shared services, ensuring that IT solutions effectively meet business needs and improve organizational efficiency. My current residence in Albany, NY, and commitment to New York State residency make me an ideal candidate. I am eager to leverage my business analysis experience and leadership capabilities to support the Finance Administration group and help agencies achieve their program goals and objectives.

Thank you for considering my application. I have attached my resume for your review and look forward to the opportunity to discuss how my business analysis experience and leadership expertise align with your team's needs.

\vspace{0.2in}

Sincerely, \\
\vspace{0.3in}
Sai Manoj Kartala

\end{document}


